\PassOptionsToPackage{dvipsnames,table}{xcolor}
\documentclass[10pt]{beamer}
\usetheme[options]{Madrid} 
\usepackage{../../latex/Cours}

\begin{document}
	
	\input{\detokenize{../../latex/MacrosCours.tex}}
\setcounter{numchap}{12}

\newcommand{\AB}{\cnum Arbres binaires de recherche}

\pythonmode

% Relation entre hauteur et taille
\begin{frame}
	\mframe{\AB}
	\begin{alertblock}{Parcours d'un arbre : Rappels}
		On peut parcourir un arbre binaire :
		\begin{itemize}[label=\textbullet]
			\item<2-> En largeur, cela revient à lister les noeuds par ordre croissant de profondeur et de gauche à droite \\
			      \onslide<3-> \textcolor{gray}{L'implémentation de ce parcours peut se faire à l'aide d'une file dans laquelle on stocke les noeuds restants à parcourir. A chaque fois qu'on traite un noeud, on le defile et on enfile ses fils (voir la fiche d'activité).}
			\item<3-> En profondeur, on tire alors partie de la structure récursive des arbres. Pour parcourir l'arbre $T=(e,sag,sad)$ on doit relancer le parcours sur $sag$ et $sad$. On distingue alors trois parcours suivant que $e$ est affiché avant, entre ou après $sag$ et $sad$ :
			      \begin{itemize}
				      \item<4-> Dans le parcours préfixé, $e$ est affiché avant de parcourir $sag$ et $sad$.
				      \item<5-> Dans le parcours infixé, $e$ est affiché après le parcours de $sag$ mais avant celui de  $sad$.
				      \item<6-> Dans le parcours suffixé, $e$ est affiché après le parcours de $sag$ et $sad$
			      \end{itemize}
		\end{itemize}
	\end{alertblock}
\end{frame}


\begin{frame}
	\mframe{\AB}
	\begin{alertblock}{Arbre binaire de recherche}
		Un arbre binaire \textcolor{blue}{de recherche} (noté {\sc abr}), est un arbre binaire tel que :
		\begin{itemize}[label=\textbullet]
			\item<2-> Les étiquettes des noeuds, appelées \textcolor{blue}{clé} sont toutes comparables entre elles. \\
			      \onslide<4->{\textcolor{gray}{Par exemple, les étiquettes sont toutes des nombres ou encore des chaines de caractères (comparées par ordre alphabétique).}}
			\item<5-> Pour tous les noeuds l'ensemble des clés présentes dans le sous arbre gauche (resp. droit) sont strictement inférieures (resp. supérieures) à la clé du noeud.\\
			      \onslide<6->{\textcolor{gray}{Par souci de simplicité, on admettra que les clés sont uniques dans un {\sc abr} ce qui permet d'éviter le cas de clés égales}}
		\end{itemize}
	\end{alertblock}
\end{frame}


\begin{frame}
	\mframe{\AB}
	\begin{alertblock}{Recherche dans un {\sc abr}}
		\begin{itemize}[label=\textbullet]
			\item<1-> La recherche d'un élément dans un {\sc abr} a pour complexité la hauteur de cet arbre. En effet, on descend d'un niveau dans l'arbre à chaque étape de la recherche en choisissant d'aller à gauche ou à droite suivante que l'élément recherché est plus petit ou plus grand que le noeud parcouru.
			\item<2-> Par conséquent, si l'arbre est dégénéré, la hauteur est égale au nombre de noeuds et l'algorithme équivaut à la recherche dans une liste.
			\item<3-> Si l'arbre est complet par contre la complexité est logarithmique et équivaut à une recherche dichotomique dans une liste triée.
		\end{itemize}
	\end{alertblock}
\end{frame}

\end{document}