\documentclass[a4paper, 11pt]{article}

\title{
  \rule{\linewidth}{0.8pt}
  \Large{\scshape{\textbf{Numérique et Science Informatique}}}

  \vspace{0.5cm}
  \large{\textit{Bac Blanc - Corrigé}}

  \rule[10pt]{\linewidth}{0.8pt}
    }
\author{}
\date{20 février 2023}

%-------------------------------------------------------
% Les packages
%-------------------------------------------------------
\usepackage[T1]{fontenc}
\usepackage[utf8]{inputenc}
% \usepackage{multicol} % les colonnes
\usepackage[french]{babel}
\addto\captionsfrench{\def\tablename{Tableau}}
\addto\captionsfrench{\def\figurename{Figure}}
\usepackage{lmodern}
\usepackage{hyperref} % liens hypertextes
% \usepackage{graphicx} % insertion d'images
\usepackage{booktabs} % Les tableaux
\usepackage{amssymb} % symboles maths
\usepackage{amsmath} % symboles maths
% \usepackage{blindtext} % faux texte
\usepackage{float} % positionnement des tableaux et autres flottants
\usepackage{multirow} % lignes fusionnées dans les tableaux 
\usepackage{lastpage} % compter les pages

%-------------------------------------------------------
% Figure geogebra -> tikz
%-------------------------------------------------------
\usepackage{pgfplots}
\pgfplotsset{compat=1.15}
\usepackage{mathrsfs}
\usetikzlibrary{arrows, positioning, snakes, calc, shapes.misc, decorations.shapes, decorations.markings, backgrounds}

%-------------------------------------------------------
% Dimension de la page
%-------------------------------------------------------
%\usepackage[a4paper, total={17cm, 25cm}, includehead, headsep=12pt]{geometry} %taille de la page et marges
\usepackage[a4paper,vmargin=1.5cm,hmargin=1.5cm,tmargin=2cm,bmargin=1.5cm]{geometry}
%-------------------------------------------------------
% Longueurs personnelles
%-------------------------------------------------------
% \setlength{\textfloatsep}{0.1cm}
\setlength{\parskip}{6pt}
\setlength{\parindent}{15pt}
% \setlength{\textfloatsep}{10pt plus 1.0pt minus 2.0pt}
% \setlength{\floatsep}{10pt plus 1.0pt minus 2.0pt}
% \setlength{\intextsep}{10pt plus 1.0pt minus 2.0pt}
\newlength{\aftercode}
\setlength{\aftercode}{20pt}

%-------------------------------------------------------
% Les entêtes et pied de pages
%-------------------------------------------------------
\usepackage{fancyhdr}
\pagestyle{fancy}
\fancyhf{}
\fancyhead[L]{\scriptsize Terminale - NSI}
\fancyhead[R]{\scriptsize Bac Blanc}
\fancyfoot[C]{\scriptsize - $\thepage$/\pageref{LastPage} -}
\renewcommand{\headrulewidth}{0.3pt}
\renewcommand{\footrulewidth}{0.3pt}

%-------------------------------------------------------
% La numérotation des questions et des listes
%-------------------------------------------------------
\usepackage[shortlabels]{enumitem}
\setlist[itemize,1]{label=\textbullet} % le label des listes de niveau 1 est un bullet
\setlist[itemize,2]{label=$\circ$} % le label des listes de niveau 2 est un bullet vide
\setlist[itemize,3]{label=$\diamond$} % le label des listes de niveau 3 est un diamant


%-------------------------------------------------------
% L'environnement Exercice
%-------------------------------------------------------
\newcounter{exercice}
\newenvironment{Exercice}[1][]{
    \refstepcounter{exercice}
    \par
    \noindent
    {\large
    \underline{\textit{Exercice~\theexercice~:}}~#1}
    % \vspace{\parskip}
    \par}
    {\vspace{1em}}

%-------------------------------------------------------
% Dossiers contenant les images
%-------------------------------------------------------
\graphicspath{ 
	{images/}
}

%-------------------------------------------------------
% L'environnement Question
%-------------------------------------------------------
\newcounter{question}[exercice]
\newenvironment{Question}{
    \refstepcounter{question}
    \begin{list}{}{%
      \setlength{\labelindent}{\parindent}
      \setlength{\leftmargin}{2\parindent}
      \setlength{\parsep}{\parskip} % Espacement vertical des sous-paragraphes
    }
    \item[\medskip\textbf{\thequestion.}]}
    {\end{list}}

%-------------------------------------------------------
% L'environnement subQuestion
%-------------------------------------------------------
\newcounter{subquestion}[question]
\newenvironment{subQuestion}{
    \refstepcounter{subquestion}
    \begin{list}{}{%
      \setlength{\leftmargin}{12pt}
      \setlength{\labelwidth}{8pt}
      \setlength{\labelsep}{4pt}
      \setlength{\listparindent}{0em}
      \setlength{\itemindent}{0em}
      \setlength{\parsep}{\parskip} % Espacement vertical des sous-paragraphes
    }
    \item[\medskip\textbf{\alph{subquestion}.}]}
    {\end{list}}
  

%-------------------------------------------------------
% Création de couleurs
%-------------------------------------------------------
\usepackage{xcolor}
\definecolor{bg_gris}{rgb}{0.95,0.95,0.95}

%-------------------------------------------------------
% Mise en forme du code
%-------------------------------------------------------
\usepackage[newfloat=false]{minted}
\setminted{autogobble, tabsize=4, breaklines, breakafter=_, linenos, numberblanklines=false, resetmargins, frame=leftline}
% Environnement python code
\newminted{python}{} % permet d'utiliser \begin{pythoncode} pour écrire du python

% Environnement mintinlinepy
\newmintinline[mintinlinepy]{python}{breaklines, breakafter=_} % permet d'utiliser \mintinlinepy{...} pour écrire du python en ligne
    
% Environnement textcode
\newminted{text}{} % permet d'utiliser \begin{textcode} pour écrire du code en français
% \BeforeBeginEnvironment{textcode}{\vspace{-2\parskip}\begin{listing}[H]}
% \AfterEndEnvironment{textcode}{\end{listing}\vspace{-\aftercode}}

% Environnement textcode
\newmintinline[mintinlinetext]{text}{breaklines, breakafter=_} % permet d'utiliser \mintinlinetext{...} pour écrire du code en ligne en français


% Environnement de code sur plusieurs pages
\usepackage{caption}
\newenvironment{longcode}{\captionsetup{type=listing}}{}

%-------------------------------------------------------
% Commandes personnelles
%-------------------------------------------------------
\newcommand{\python}{\texttt{python}}
% \renewcommand{\thesection}{\Alph{section}.}
%-------------------------------------------------------

% Le document
%-------------------------------------------------------
\begin{document}

%-------------------------------------------------------
% La page de titre
%-------------------------------------------------------
\maketitle
\thispagestyle{fancy}

\begin{Exercice}[\textit{Base de données} \hfill 8 points]

  \begin{Question}
    La table \mintinline{SQL}{Articles} utilise des clés étrangères des tables 
    \mintinline{SQL}{Auteurs} et \mintinline{SQL}{Themes}. Si ces dernières sont vides, il n'est pas possible de lier
    les valeurs et donc on ne peut pas insérer de valeurs.
  \end{Question}

  \begin{Question}
    On saisit \mintinline{SQL}{INSERT INTO Traitements (article, theme) VALUES  (2, 4)}
\end{Question}

\begin{Question}
    On saisit \mintinline{SQL}{UPDATE Auteurs SET nom = "Jèraus" WHERE idAuteur = 2}
  \end{Question}

  \begin{Question}
    \begin{subQuestion}
        Le titre des articles parus après le $1^{er}$ janvier 2022 inclus :
        \begin{minted}{SQL}
SELECT titre 
FROM Articles
WHERE dateParution >= 20220101
        \end{minted}
    \end{subQuestion}
    \begin{subQuestion}
        Le titre des articles écrits par l'auteur Étienne Zola :
        \begin{minted}{SQL}
SELECT titre
FROM Articles
WHERE auteur = 3
        \end{minted}
    \end{subQuestion}
    \begin{subQuestion}
        Le nombre d'articles écrits par l'auteur Jacques Pulitzer (présent dans la table \mintinline{SQL}{Auteurs} mais on ne connaît pas son \mintinline{SQL}{idAuteur}) :
        \begin{minted}{SQL}
SELECT count(*)
FROM Articles
JOIN auteurs ON Articles.auteur = Auteurs.idAuteur
WHERE nom LIKE "Pulitzer" AND prenom LIKE "Jacques"
        \end{minted}
    \end{subQuestion}
    \begin{subQuestion}
        Les dates de parution des articles traitant du thème \og Sport \fg\:
        \begin{minted}{SQL}
SELECT date
FROM Articles
JOIN Traitements ON Articles.idArticle = Traitements.article
JOIN Themes ON Traitements.theme = Themes.idTheme
WHERE Themes.themes LIKE "Sport"
        \end{minted}
    \end{subQuestion}
  \end{Question}

\end{Exercice}

\newpage

\begin{Exercice}[\textit{Arbres binaires équilibrés} \hfill 12 points]

  \noindent{\textbf{Partie A :}}
  \setcounter{question}{0}

  \begin{Question}
    \begin{subQuestion}
      On obtient :

      \begin{figure}[H]
        \begin{center}
          \footnotesize{
            \begin{tikzpicture}[level distance=1cm,level 1/.style={sibling distance=4cm},
                level 2/.style={sibling distance=2cm},
                level 3/.style={sibling distance=1.5cm}]
              \tikzstyle{every node}=[draw, circle, minimum size=7mm, inner sep=1mm]
    
              \node at (0,0) (9) {9} []
              child {node (6) {6}
                  child {node (3) {3}
                      child {node (2) {2}}
                      child {node[draw=none] {} edge from parent [draw=none]}
                    }
                  child {node (7) {7}
                    }
                }
              child {node (15) {15}
                  child {node[draw=none] {} edge from parent [draw=none]}
                  child {node (18) {18}
                      child {node (16) {16}}
                      child {node[draw=none] {} edge from parent [draw=none]}
                    }
                };

                \node[draw=none, xshift=-6mm, yshift=0mm] at (9) {0};
                \node[draw=none, xshift=-6mm, yshift=0mm] at (6) {-1};
                \node[draw=none, xshift=-6mm, yshift=0mm] at (15) {2};
                \node[draw=none, xshift=-6mm, yshift=0mm] at (3) {-1};
                \node[draw=none, xshift=-6mm, yshift=0mm] at (7) {0};
                \node[draw=none, xshift=-6mm, yshift=0mm] at (18) {-1};
                \node[draw=none, xshift=-6mm, yshift=0mm] at (2) {0};
                \node[draw=none, xshift=-6mm, yshift=0mm] at (16) {0};
    
            \end{tikzpicture}
          }
        \end{center}
      \end{figure}
    \end{subQuestion}
    \begin{subQuestion}
      Cet arbre n'est pas équilibré car le nœud de valeur 15 a une balance de 2.
    \end{subQuestion}
  \end{Question}


  \begin{Question}
    \begin{subQuestion}
      On obtient \mintinlinepy{[0, 45, 40, 48, 17, 43, 46, 49, 14, 19]}
    \end{subQuestion}

    \begin{subQuestion}
      On obtient :

      \begin{figure}[H]
        \begin{center}
          \footnotesize{
            \begin{tikzpicture}[level distance=1cm,level 1/.style={sibling distance=4cm},
                level 2/.style={sibling distance=2cm},
                level 3/.style={sibling distance=1.5cm}]
              \tikzstyle{every node}=[draw, circle, minimum size=7mm, inner sep=1mm]
    
              \node at (0,0) {35} []
              child {node {30}
                child {node {20}
                  child {node {2}}
                  child {node[draw=none] {} edge from parent [draw=none]}
                }
                child {node {23}}
              }
              child {node {12}
                child{node {10}}
                child{node {5}}
              };
            \end{tikzpicture}
          }
        \end{center}
      \end{figure}   
    \end{subQuestion}
  \end{Question}


  \begin{Question}
    \begin{subQuestion}
      \mintinlinepy{f(arbre, 1)} renvoie 3. En effet, si l'arbre est vide ou si la valeur de sa racine est \mintinlinepy{None}, on renvoie $0$. Dans le cas contraire, on renvoie
      $1$ plus de maximum des résultats des sous-arbres gauches et droits (indices \mintinlinepy{2*i} et \mintinlinepy{2*i+1}). On calcule ainsi la hauteur de l'arbre.
    \end{subQuestion}
    \begin{subQuestion}
      La fonction \mintinlinepy{f} permet de calculer la hauteur d'un arbre.
    \end{subQuestion}
  \end{Question}

  \begin{Question}
    \begin{minted}[]{python}
def estEquilibre(arbre: list, i : int) -> bool:
   if i >= len(arbre) or arbre[i] is None:
        return True
    else:
        balance = f(arbre, 2*i+1) - f(arbre, 2*i)
        reponse = balance in [-1, 0, 1]
        return reponse and estEquilibre(arbre,2*i) and estEquilibre(arbre, 2*i+1)
\end{minted}
  \end{Question}

  \noindent{\textbf{Partie B :}}
  \setcounter{question}{0}

  \begin{Question}
    \begin{itemize}
      \item Parcours \textit{préfixe} : 45, 40, 17, 14, 19, 43, 48, 46, 49
      \item Parcours \textit{infixe} : 14, 17, 19, 40, 43, 45, 46, 48, 49 
      \item Parcours \textit{suffixe} : 14, 19, 17, 43, 40, 46, 49, 48, 45
    \end{itemize}
  \end{Question}

  \begin{Question}
    On obtient :

      \begin{figure}[H]
        \begin{center}
          \footnotesize{
            \begin{tikzpicture}[level distance=1cm,level 1/.style={sibling distance=4cm},
                level 2/.style={sibling distance=2cm},
                level 3/.style={sibling distance=1.5cm}]
              \tikzstyle{every node}=[draw, circle, minimum size=7mm, inner sep=1mm]
    
              \node at (0,0) {23}
                child {node {21}
                  child {node {20}}
                  child {node {22}}
                }
                child {node {25}
                  child {node {24}}
                  child {node[draw=none] {} edge from parent [draw=none]}
                };
            \end{tikzpicture}
          }
        \end{center}
      \end{figure}   
  \end{Question}
  
  \begin{Question}
    On propose :
\begin{minted}{python}
def infixe(arbre: list) -> list:
    pile = []
    visites = []
    n = 1
    repetition = True
    while repetition :
        while n < len(arbre) and arbre[n] is not None :
            pile.append(n)
            n = 2*n
        if len(pile) == 0 :
            repetition = False
        else :
            n = pile.pop()
            visites.append(arbre[n])
            n = 2*n+1
    return visites
\end{minted}
  \end{Question}

  \begin{Question}
    On propose :
\begin{minted}{python}
def construitABR(i, ordre):
    while len(nouveau) <= i:
        nouveau.append(None)

    i_milieu = len(ordre)//2
    nouveau[i] = ordre[i_milieu]
    
    gauche = ordre[:i_milieu]
    if len(gauche) > 0:
        construitABR(2*i, gauche)
    
    droite = ordre[(i_milieu+1):]
    if len(droite) > 0:
        construitABR(2*i+1, droite)
\end{minted}
  \end{Question}

\end{Exercice}

\end{document}